% !TEX root = SocialVision2012.tex
\Section{Background and Related Work}\label{sec:background}

We will build on computer vision, sociology, and social network analysis. This section discusses the relevant background in each and some early explorations of their intersection, and space restrictions prohibit an exhaustive discussion. %The background in this section provides a foundation for our proposed research in Sect.~\ref{sec:proposed-research}.

\boldstart{Existing tools from computer vision}

\emph{1) Facial analysis and identity recognition}. One set of practical technologies that we will leverage include those for detecting faces~\cite{ViolaJones,Zhang:detect}; tracking them (e.g.~\cite{Comaniciu:track}); and extracting information about pose (e.g.,~\cite{Hanson,Murphy-Chutorian:pose}), expression (e.g.,~\cite{Matthews:AAM,Lucey:AAM,Mumford:face,Yacoob:expression,delaTorre:expression,Essa:expression}), identity (e.g.,~\cite{Chellappa:face}), as well as gender, age, ethnicity (e.g.~\cite{LNCS53050340}).  Current systems are imperfect, but their performance is at useful levels thanks to the explosion of personal and community photo collections, the introduction of social tagging~\cite{Stone2008,Stone2010} and other (semi-)automatic tagging mechanisms~\cite{berg2004naf,berg2005sp,Everingham06a,huang:lfw,YangBKR12}, and public demand for applications. A recent test by PI Zickler and colleagues showed that a relatively simple computational pipeline is able to achieve close to 90\% accuracy on a 100-way identity recognition task when sixty or more annotated facial samples are available for each of the one-hundred individuals in the gallery~\cite{PintoZickler2011}. We will leverage these technologies in two distinct ways. First, we will use estimates of facial location, pose, and expression (when possible) as sources of information about the social interactions. Second, we will use estimates of identity to associate these interactions with individuals (``nodes'') in the underlying social network. \comment{As stated in the introduction, one of the main challenges that we will address is how to deal with the uncertainty that is inherent to these sources of information. }


%%%%%%%%%%%%%%%%%%%%%%%%%%%%%%%%%%%%%%%%%


\emph{2) Body analysis and individual activity recognition}. We will also leverage existing technology for detecting individual bodies or  parts~\cite{Dalal:HOG,poselet,pose_part}, tracking them over time~\cite{RamananFZ07,EshelM10}, and extracting information about pose, gestures, and other fine-scale movements~\cite{Mitra:gesture,Ryoo:action,Poppe}. Practical techniques also exist for a variety of image quality levels, from high-resolution spatio-temporal descriptors~\cite{Dollar:STIP,Laptev:STIP,Brox:flow} and activity grammars~\cite{Niebles2007,Niebles2006} to very coarse histograms-of-flow in low-resolution video~\cite{EfrosBMM03}. These descriptors have proven useful for detecting pre-defined action categories in the presence of clutter (e.g.,~\cite{,groupdet2013, Li2010} by PIs Li and Zickler), and for discriminating individual action categories (e.g., walking, running, hand waving)~\cite{Weizmann,KTH} even under substantial changes in camera position (e.g.,\cite{Weinland:invariance2}, and ~\cite{LiZickler2012} by PIs Li and Zickler). Our goal is to use similar descriptors as a starting point of a different problem: detecting and recognizing social proxemes involving two or more individuals. 

%Put this somewhere else:  Related tasks also include detecting motion from clutter \cite{Li2010}, comparing two behaviors \cite{LiPAMI 2012}, as well as adapting behavior representations between different modalities \cite{LiZickler2012,Li2011}, which are all among co-PI Li's expertises.


%%%%%%%%%%%%%%%%%%%%%%%%%%%%%%%%%%%%%%%%%

\boldstart{Existing tools from sociology and network analysis}

\emph{1) Qualitative and quantitative sociology}. Despite of the lack of automation, sociology has been under investigations for decades \cite{Darwin,Thomkins,Goffman,Kendon1990,Ekman,Hoyle,Tannen}. Interactions, proxemes, and relations\comment{among socialized individuals} can be studied qualitatively or quantitatively \cite{hall1974,Goodwin2000,Goldin,Goodwin2007,Kendon2010}, while contemporary communication and internet enables analytical and statistical approaches\comment{that can be} applied to new socializing modalities such as emails \cite{Eckmann} and mobile phones calls \cite{Onnela,Eagle}. Communities can be abstracted from various domains including education \cite{Scherr2009}, production \cite{Watts}, transportation \cite{Gonzalez},  and politics \cite{Iacus}. All together has shaped the interdisciplinary study on `computational social science' \cite{Lazer2009} and `social signal processing' \cite{Pantic}. These research will benefit our research by either providing proxemes to exploit or inspiring us to learn/discover new proxemes. Meanwhile, our expected outcomes will automate proxeme detection and discovery in new environments and for large group sizes over long periods, which is prohibitive for manual work adopted in current sociological studies.

%These research will benefit socially-aware computer vision by providing categories of socially-meaningful interactions --- proxemes --- to exploit, such as `f-formations' \cite{Kendon1990} which describes the spatial proximity among participants of a short-term interaction. Meanwhile, our proposed research on proxeme detection and discovery will allow learning of their categories in new environments and for large group sizes over long periods, which is prohibitive for manual work adopted in current sociological studies.


%%%%%%%%%%%%%%%%%%%%%%%%%%%%%%%%%%%%%%%%%

\emph{2) Computational network models}. Probabilistic mechanisms for depicting the dynamics of a small-scale group are available in \cite{Basu:meeting,Dong,Choudhury:MHMM,Pan:influence}, and our research also relates to graph cut \cite{Ng:spectral,Boykov:segmentation}, clustering \cite{Filippone:clustering,Xu:clustering}, and graph matching \cite{West:Graph,Caetano:graph}. Statistical network models\comment{as well as a diversity of problems regarding learning and inference on graphs} are particularly useful in characterizing a broad body of network phenomena (e.g., message propagation and small-world effect) and solving a diversity of network problems (e.g. link prediction and page recommendation). We refer the reader to \cite{Goldenberg,Kolacyzk,Snijders,Rossi} for comprehensive coverage of representative work\comment{on statistical models of networks}. In particular, research has considered the `multi-view' network representation, where a view may refer to \comment{a type of low-level observation or source that characterizes the affinities between nodes, such as citations between papers or the similarity of the words/topics in two papers \cite{ChangB09,WangMM05}, and a view can also refer to }a type of high-level social role or membership \cite{AiroldiBFX08,Kim12}. Missing links or nodes are common for realistic networks as accounted for by network completion \cite{Clauset,Guimera,HannekeX09,KimL11}, and new links will be emerging over time as studied by link prediction \cite{Goldberg,Liben-Nowell,TaskarWAK03}. A visually sensed social network must consider these conditions together and confront new challenges of inferring high-level relational semantics from high-dimensional multi-modal image features and missing/noisy data. These will be our proposed research problems in Section \ref{sec:vis2net}.


%%%%%%%%%%%%%%%%%%%%%%%%%%%%%%%%%%%%%%%%%

\boldstart{Recent Influences}

While computer vision and social network analysis have mostly been studied separately, there are some notable predecessors that successfully united them for certain tasks in certain domains.

\emph{1) Enhancing visual recognition with social context}. With the growth of online photo collections, an emerging sub-field aims to use social network information as context to improve image-based recognition.\comment{For the most part, these methods rely explicitly on social network information extracted from ``Facebook friendships'' and other Internet-based relationship information, and they do not attempt to recover social information from the images themselves.} PI Zickler has made the pilot contribution in face recognition using social contexts \cite{Stone2008,Stone2010}. Similar attempts include exploiting clothing that people wear over the course of a few hours or a day~\cite{anguelov2007cir, zhang2003aah,  song2006cah, sivic2006fpr}, using temporal and spatial context of photographs~\cite{naaman2005lcr, zhao2006apa}, or using the frequency with which people appear together~\cite{anguelov2007cir}, and they spurred the problem of joint scene classification for many images \cite{McAuley:socialclassify} and other recent results \cite{Dikmen:classify,LeeBMVC2011,Poppe2012,occupation2013}. These methods solve a problem that is opposite to ours in that they use social network information from non-image sources to improve computer vision. We seek to do the opposite\comment{by using computer vision to infer social network information front he images themselves}, with the belief that this information will complement the social information available from other non-image sources. These efforts also provide an application of our methods: Once images are used to recover social network information using our methods, these social contexts can be used in turn to improve computer vision. This feedback also suggests unsupervised approaches that simultaneously optimize visual and social analysis, as one of our long-term goals (see Section~\ref{sec:closeloop}).

%%%%%%%%%%%%%%%%%%%%%%%%%%%%%%%%%%%%%%%%%

\emph{2) Inferring elementary social roles from Imagery}. Seminal work has only appeared recently to infer elementary social attributes in each image such as `parents-children'\comment{from simple geometric configurations and appearance descriptors of faces or body parts} \cite{Gallagher,Wang2010,Murillo2012}, without connecting the image targets to nodes in the social network. `Alliance' clusters over the movie characters can be inferred from co-occurrences and shot analysis \cite{Ding2010,Ding2011}, the group leader can also be agglomerated from surveillance \cite{Yu2009,Zhang2011}, and roles in pre-defined social events (\emph{e.g.}, meetings \cite{meetingrolerecognition}, sport games \cite{LanSM12}, and weddings \cite{FeiFeiRole2013}) can be estimated within each video. These preliminary work are simplified cases of our expected outcome, which aims to explicitly mapping targets into nodes in the network of much larger scale. Instead of leveraging a few simple visual cues for inferring a few concepts, we will extract socially informative mid-level representations -- proxemes, aggregate repetitive occurrences of the proxemes from photo collections or surveillance networks \cite{CamNetRoy,CamNetSclaroff}, and develop systematic machines to estimate multi-view affinities between social members.


\emph{3) Multi-people activity analysis}. More related to our work are the predecessors in analyzing behaviors involving more than one person. Many recognize a group activity that is temporally cropped around its occurrence and where all targets are participating \cite{Intille:act,Ni:group,Lan:Group,Patron-PerezMRZ12,PrabhakarR12}, while some consider a salient activity to span only a short duration in a long video \cite{Hongeng:act,Hakeem:act,Ba:meeting,McCowan:meeting,CHIL, Choi:recogtrack, Ryoo:group, Regh2013}. PIs Li's and Zickler's work\cite{groupdet2013,LiIJCV2012,Li2010} are among the first set of work to distinguish salient activities from social groups. However, the most generic case we propose to address, is generic socially-informative proxemes both embedded in clutter and temporally localized, as a generalization of detecting `f-formation'~\cite{Cristani:fformation} and a counterpart to socially non-meaningful collective behaviors~\cite{Amer:group}.


%The earliest efforts were devoted to analyzing indoor meetings involving handfuls of participants \cite{GaticaPerez,McCowan:meeting}, where descriptors derived from tracking \cite{Smith:track} and pose estimation \cite{Ba:meeting} have been integrated in dynamic hidden Markov models (HMM)~\cite{Zhang:meeting}\todd{need more here. What was the output of their system? How is this different or similar to us? Has any work on meetings defined interaction categories, or recovered social network information from occurrences of these categories?}. Recent efforts have explored more complex visual environments, including the detection and recognition of a small number of pre-defined two-person interaction categories (e.g., hand-shaking, pushing, hugging)~\cite{UTdata}\todd{More here. Cite the related previous work in the CVPR submission.}; the detection of group conversations in a cocktail-party scenario; the detection of certain pre-defined categories of collective large-group behavior~\cite{Choi:context,Choi:recogtrack,Amer:group,Lan:Group}; and PI Li's work on segmenting offensive and defensive players in sports matches~\cite{LiIJCV2012}. The promising results from this assortment of sub-problems is part of what motivates us to pursue a large-scale effort to unify social and visual analysis more broadly, jointly considering all aspects of socialization in image and video data. \todd{It is important that we do not miss any related work in this paragraph. We need to be careful to include everything.}
