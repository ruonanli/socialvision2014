% !TEX root = SocialVision2014.tex
\Section{Background and Related Work}\label{sec:background}

We will build on ideas from semantic image processing, social sciences, network analysis, and signal processing on graphs. This section discusses the relevant background in each, as well as some early explorations of their intersection.
%, and space restrictions prohibit an exhaustive discussion. %The background in this section provides a foundation for our proposed research in Sect.~\ref{sec:proposed-research}.

\boldstart{Semantic Image and Video Processing}. We will leverage existing semantic image processing technology for detecting faces~\cite{ViolaJones,Zhang:detect}, tracking them (e.g.,~\cite{Comaniciu:track}), and extracting pose (e.g.,~\cite{Hanson,Murphy-Chutorian:pose}), expression (e.g.,~\cite{Matthews:AAM,delaTorre:expression,Essa:expression}), and other attributes like gender, age, and ethnicity (e.g.,~\cite{LNCS53050340, Gender, Age}).  In particular, we will employ approaches for face verification \cite{ChopraVerification,Kumar09attributeand,DeepFace} to compute similarity scores between faces. These technologies are imperfect, but performance is at useful levels thanks to exploding photo collections, public demand for applications, and the emergence of social tagging~\cite{Stone2008,Stone2010} and other tagging mechanisms~\cite{berg2004naf,berg2005sp,Everingham06a,huang:lfw,YangBKR12}. We will leverage these technologies in two distinct ways. First, we will use estimates of facial location and pose as features for representing interactions. Second, we will use (noisy) estimates of identity similarities between faces and other attributes to associate the faces with individuals in the underlying social network. 

We will also leverage existing technology for detecting human bodies and body parts~\cite{Dalal:HOG,poselet,pose_part}, tracking them~\cite{RamananFZ07,EshelM10}, and extracting pose, movements, and gestures (when possible)~\cite{Mitra:gesture,Ryoo:action,Poppe}. There are appropriate technologies for a variety of image qualities, from high-resolution spatio-temporal descriptors~\cite{Dollar:STIP,Laptev:STIP,Brox:flow} and activity grammars~\cite{Niebles2007,Niebles2006} to very coarse histograms-of-flow for low-resolution video~\cite{EfrosBMM03}. They produce descriptors proven to be good enough for detecting pre-defined action categories in the presence of clutter (e.g.,~\cite{groupdet2013, Li2010} by PIs Li and Zickler), and for discriminating between individual action categories (e.g., walking, running), even under substantial changes in camera position (e.g.,\cite{Weinland:invariance2}, and~\cite{LiZickler2012} by PIs Li and Zickler). We will use these descriptors as a starting point for representing, detecting, and recognizing proxemes, and for analyzing interactions.

We will build directly on recent progress toward recognizing interactions between multiple individuals in groups. A number of methods recognize particular categories of group-interactions, such as hugs and hand-shakes, in video segments that are manually pre-cropped in time and contain no distracting, non-participating bystanders~\cite{Intille:act,Patron-PerezMRZ12,PrabhakarR12}, while others focus on salient group-interactions that span only short durations within longer videos~\cite{Hongeng:act,Hakeem:act,Choi:recogtrack, Ryoo:group, Regh2013}. In Refs.~\cite{groupdet2013,LiIJCV2012,Li2010}, PIs Li and Zickler were among the first  to develop methods  for detecting and localizing general, salient group-interactions that are embedded in longer videos of larger groups (others are \cite{Cristani:fformation,Amer:group}). The  relevant parts of this previous work are detailed in Sec.~\ref{sec:activity}.


\boldstart{Influence from Social Science and Social Signal/Image Processing}. We are inspired by qualitative studies of non-verbal communications, and our use of the term \emph{proxeme} is inspired by anthropologist Edward T.~Hall, who coined it as analogous to the phoneme of language~\cite{hall1974}. Our goals of learning and detecting proxemes can be seen as automating the intense manual process of \emph{Context Analysis} developed by sociologists, anthropologists, and psychiatrists~\cite{Kendon1990}, where an expert watches and re-watches filmed specimens of interaction, identifies the repertoire of ``behavioral units'' customary in that setting, and identifies the quasi-grammatical rules by which these units are organized, as well as how these units are modulated by roles and relationships. We are also inspired by recent emergence of the inter-disciplinary areas of \emph{computational social science} \cite{Lazer2009,Pantic}, made possible by the increasing availability of digitized records of textual and verbal communication. These work motive ways to improve image recognition by incorporating social context: PI Zickler pioneered face recognition in online photos using social network context~\cite{Stone2008,Stone2010}, which has since been improved by others~\cite{Dikmen:classify,Poppe2012,LeeBMVC2011,hanalbum2013album}. \comment{This effort is corroborated by that of using social context to enhance scene recognition~\cite{McAuley:socialclassify} and occupation recognition~\cite{occupation2013}.} Whereas these methods use social information from non-image sources to improve image analysis, we seek to do the opposite, with the belief that visual information will complement social network information available from other non-image sources. 

We also build on recent success in inferring small numbers of pre-determined social roles and relationships from images and video, such as parent-child relationships in image collections~\cite{Gallagher,Wang2010,Murillo2012}; alliance clusters based on co-occurrences and shot analysis in movies~\cite{Ding2010,Ding2011}; leader-follower relationships in surveillance video~\cite{Yu2009,Zhang2011}; brides and grooms at weddings~\cite{FeiFeiRole2013}; player positions in football~\cite{LanSM12}; and speakers and listeners in meetings~\cite{meetingrolerecognition}. We seek to build on these successes by going beyond limited roles and event types between isolated groups in single images, automatically discovering the behavioral units that are relevant to different event types, and mapping actors to nodes in social networks that are multi-view and contains tens or hundreds of individuals.


\boldstart{Signal Processing on Graphs}. Networks, with social networks as the kind of special interest in this proposal, has been extensively studied for their practical applications to characterizing information transmission effect, analyzing population dynamics, discovering structural properties, and so on \cite{Jackson:2008,David:2010}. In statistical machine learning research, one is interested in random graph models for network completion~\cite{Clauset,Guimera,HannekeX09,KimL11}, link prediction~\cite{Goldberg,Liben-Nowell,TaskarWAK03}, as well as representations for mixed-membership networks\comment{ of social members clustered in different groups of different memberships}~\cite{AiroldiBFX08,Kim12}. Nevertheless, when considering multi-type relationships as functionals defined on every edge of the social graph, the most relevant theoretical fundamentals are rooted in spectral graph theory \cite{SpectralChung}, discrete calculus \cite{Grady10}, as well as algebraic signal processing theory \cite{ASP}.

Spectral representation of a graph is at the heart of this framework \cite{Ng:spectral,Luxburg,zhang2008multiway}, where the spectral embedding of the graph nodes directly imply graph partitioning or community cluttering \cite{PhysRevE,Boykov:segmentation}. This representation also provides intrinsic geometric regularizations to data points as graph nodes in semi-supervised or supervised learning problems \cite{Zhu2005,Smola2003,Zhou04}. More recently, it has been more widely recognized that the spectral representation provides a foundation to the ``frequency-domain" analysis of signals defined on the graph nodes (or generic discrete domains) instead of in Euclidean space, and leads to the prosperity of the emerging area of signal processing on graphs \cite{MouraSurvey,ShumanSurvey}. In particular, Fourier transform, Fourier domain analysis, as well as filter design schema may be developed for scalar signals defined on graph nodes \cite{hammond,Agaskar,Rabbat,shuman_ACHA_2013}, which has been followed by the exploration of wavelets defined on the graphs \cite{CK03,GavishNC10,narang2009lifting,RamEladCohen,NIPS2013_5046,Leonardi,ShumanFV13}, in companion to the investigation on parametric filter design in the vertex domain \cite{SandryhailaFilter,hammond} and dictionary learning for graph signals \cite{ThanouSF14,ZhangDF12}. Fourier and wavelet analysis of the graph signals are also proved to add new insights to traditional unsupervised \cite{Tremblay2014} or (semi-)supervised learning \cite{shuman2011semi,narang2013localized,Ekambaram,sandryhaila2013classification,ChenTSP2014}.

Most related work to the goal of this research include graph coarsening \cite{LafonPAMI,RonCoarsening}, by which we will construct the social graph from the image targets. They also include the different formulation of filtering on the graph as energy minimization problems \cite{Grady10}, which often relate to signal diffusion from one node to the adjacent nodes on the graph/lattice \cite{Geman_diffusion,Black_diffusion,Perona_diffusion,Bouman_diffusion,Zhangdiffusion}. However, a significant feature of our multi-dimensional signals -- the multi-type relationships -- are defined on the edges instead of nodes, and thus substantially distinguishes our challenges from those addressed in the majority of literature, for which we must seek stronger tools and deeper foundations (e.g., from \cite{Grady10}). These challenges are not limited the necessity of developing novel models for signals on entities beyond nodes: Both the task of graph coarsening and relationship estimation must account for the noisy and missing inputs of proxemes and others from semantic image processing, and we will discuss them shortly in subsequent sections.

%,Lucey:AAM,
%,Mumford:face,Yacoob:expression
%identity (e.g.,~\cite{Chellappa:face})
%A recent test by PI Zickler and colleagues showed that a relatively simple computational pipeline is able to achieve close to 90\% accuracy on a 100-way identity recognition task when sixty or more annotated facial samples are available for each of the one-hundred individuals in the gallery~\cite{PintoZickler2011}. 

%\boldstart{Body analysis and behavior recognition}. 


%Put this somewhere else:  Related tasks also include detecting motion from clutter \cite{Li2010}, comparing two behaviors \cite{LiPAMI 2012}, as well as adapting behavior representations between different modalities \cite{LiZickler2012,Li2011}, which are all among co-PI Li's expertises.

%\boldstart{Video-based multi-agent activity Recognition}. 



%,Ni:group,Lan:Group,
%Ba:meeting,CHIL,McCowan:meeting, 

%%%%%%%%%%%%%%%%%%%%%%%%%%%%%%%%%%%%%%%%%



%This includes probabilistic models for analyzing discourse and interactional dynamics of small groups based on speech and text~\cite{Basu:meeting,Dong,Choudhury:MHMM,Pan:influence,Grosz:1986,Moore:discoproc,Webber,GeeBook}, 

%This general approach has contributed to studies in interactions (and thus roles and relationships) at parties~\cite{Kendon1990}, in counseling~\cite{erickson1982counselor}, in classrooms~\cite{mcdermott1978criteria}, in college tutorials~\cite{Scherr2009}, among others. A salient fact that emerges from these studies is that the proxemes in each setting are very distinct, so that knowledge gained in one setting provides only limited insight about another. By automating this process as much as possible, we will establish a framework for data-driven analysis of interactions and relationships at unprecedented scales.

%We are inspired by qualitative studies of non-verbal communication by social scientists, and our use of the term \emph{proxeme} is inspired by anthropologist Edward T.~Hall, who coined it as analogous to the phoneme of language~\cite{hall1974}. Our goals of learning and detecting proxemes can be seen as automating the intense manual process of \emph{Context Analysis} developed by sociologists, anthropologists, and psychiatrists during the second half of the twentieth century~\cite{Kendon1990}, where an expert watches and re-watches filmed specimens of interaction in a particular setting, identifies the repertoire of ``behavioral units'' customary in that setting, and identifies the hierarchical, quasi-grammatical rules by which these units are organized. Through decades of work, social scientists have used this general approach to identify behavioral units and productively analyze interactions (and thus roles and relationships) at parties~\cite{Kendon1990}, in counseling interviews~\cite{erickson1982counselor}, in first-grade classrooms~\cite{mcdermott1978criteria}, in college STEM tutorials~\cite{Scherr2009}, and more. A salient fact that emerges from these studies is that the behavioral units---or proxemes as we call them here---in each setting are very distinct, so that knowledge gained in one setting provides only limited insight about another. By automating this process as much as possible, we will establish a framework for data-driven analysis of visual interactions and relationships at unprecedented scales.


%; as well as to enhance identity recognition using clothing cues~\cite{anguelov2007cir, zhang2003aah,  song2006cah, sivic2006fpr}, time-stamps and geo-tags~\cite{naaman2005lcr, zhao2006apa}, and frequencies in which individuals appear together~\cite{anguelov2007cir}. 


%Despite of the lack of automation, sociology has been under investigations for decades \cite{Darwin,Thomkins,Goffman,Kendon1990,Ekman,Hoyle,Tannen}. Interactions, proxemes, and relations\comment{among socialized individuals} can be studied qualitatively or quantitatively \cite{hall1974,Goodwin2000,Goldin,Goodwin2007,Kendon2010}, while 


 
 
 %Our efforts can be seen as adding visual interactions as a new signal for these types of analyses, and to achieve our goals we will address the new challenges of handling uncertainty in inferred identities and appropriately distilling high-dimensional, multi-cue, noisy visual features.

%, and our research also relates to graph cut \cite{Ng:spectral,Boykov:segmentation}, clustering \cite{Filippone:clustering,Xu:clustering}, and graph matching \cite{West:Graph,Caetano:graph}. 

% statistical network models for analyzing connections and dynamics in much larger groups~\cite{Goldenberg,Kolacyzk,Snijders,Rossi}.

% enables analytical and statistical approaches\comment{that can be} applied to new socializing modalities such as emails \cite{Eckmann} and mobile phones calls \cite{Onnela,Eagle}. Communities can be abstracted from various domains including education \cite{Scherr2009}, production \cite{Watts}, transportation \cite{Gonzalez},  and politics \cite{Iacus}. 


%Contemporary communication and internet enables analytical and statistical approaches\comment{that can be} applied to new socializing modalities such as emails \cite{Eckmann} and mobile phones calls \cite{Onnela,Eagle}. Communities can be abstracted from various domains including education \cite{Scherr2009}, production \cite{Watts}, transportation \cite{Gonzalez},  and politics \cite{Iacus}. All together has shaped the interdisciplinary study on `computational social science' \cite{Lazer2009} and `social signal processing' \cite{Pantic}. These research will benefit our research by either providing proxemes to exploit or inspiring us to learn/discover new proxemes. Meanwhile, our expected outcomes will automate proxeme detection and discovery in new environments and for large group sizes over long periods, which is prohibitive for manual work adopted in current sociological studies.

%These research will benefit socially-aware computer vision by providing categories of socially-meaningful interactions --- proxemes --- to exploit, such as `f-formations' \cite{Kendon1990} which describes the spatial proximity among participants of a short-term interaction. Meanwhile, our proposed research on proxeme detection and discovery will allow learning of their categories in new environments and for large group sizes over long periods, which is prohibitive for manual work adopted in current sociological studies.

%%%%%%%%%%%%%%%%%%%%%%%%%%%%%%%%%%%%%%%%%

%\boldstart{Computational network analysis}. Probabilistic mechanisms for depicting the dynamics of a small-scale group are available in \cite{Basu:meeting,Dong,Choudhury:MHMM,Pan:influence}, and our research also relates to graph cut \cite{Ng:spectral,Boykov:segmentation}, clustering \cite{Filippone:clustering,Xu:clustering}, and graph matching \cite{West:Graph,Caetano:graph}. Statistical network models\comment{as well as a diversity of problems regarding learning and inference on graphs} are particularly useful in characterizing a broad body of network phenomena (e.g., message propagation and small-world effect) and solving a diversity of network problems (e.g. link prediction and page recommendation). We refer the reader to \cite{Goldenberg,Kolacyzk,Snijders,Rossi} for comprehensive coverage of representative work\comment{on statistical models of networks}. In particular, research has considered the `multi-view' network representation, where a view may refer to \comment{a type of low-level observation or source that characterizes the affinities between nodes, such as citations between papers or the similarity of the words/topics in two papers \cite{ChangB09,WangMM05}, and a view can also refer to }a type of high-level social role or membership \cite{AiroldiBFX08,Kim12}. Missing links or nodes are common for realistic networks as accounted for by network completion \cite{Clauset,Guimera,HannekeX09,KimL11}, and new links will be emerging over time as studied by link prediction \cite{Goldberg,Liben-Nowell,TaskarWAK03}. A visually sensed social network must consider these conditions together and confront new challenges of inferring high-level relational semantics from high-dimensional multi-modal image features and missing/noisy data. These will be our proposed research problems in Section \ref{sec:vis2net}.


%%%%%%%%%%%%%%%%%%%%%%%%%%%%%%%%%%%%%%%%%

%\subsection{Recent influences in socially-informed image processing}
%\label{sec:recent-influences}

%While vision and social network analysis have been studied separately for the most part, there are some notable predecessors that successfully unite them for certain tasks and in certain domains.

%\boldstart{Enhancing visual recognition with social context}. With the growth of online photo collections, an emerging area aims to solve the problem opposite to ours: using social network information as context to improve visual recognition. PI Zickler pioneered face recognition using social context~\cite{Stone2008,Stone2010}, which has since been improved by others~\cite{Dikmen:classify,Poppe2012,LeeBMVC2011,hanalbum2013album}. The vision community has also developed related tools for using social context to enhance scene recognition~\cite{McAuley:socialclassify} and occupation recognition~\cite{occupation2013}; as well as to enhance identity recognition using clothing cues~\cite{anguelov2007cir, zhang2003aah,  song2006cah, sivic2006fpr}, time-stamps and geo-tags~\cite{naaman2005lcr, zhao2006apa}, and frequencies in which individuals appear together~\cite{anguelov2007cir}. Whereas these methods use contextual information from non-image sources to improve computer vision, we seek to do the opposite, with the belief that visual information will complement the social network information that is available from other, non-image sources. 
%These efforts also provide an application of our methods: Once images are used to recover social network information using our methods, these social contexts can be used in turn to improve computer vision. This feedback also suggests unsupervised approaches that simultaneously optimize visual and social analysis, as one of our long-term goals (see Section~\ref{sec:closeloop}).

%%%%%%%%%%%%%%%%%%%%%%%%%%%%%%%%%%%%%%%%%

%\boldstart{Inferring elementary social roles and relationships from imagery}. We are inspired by the success of recent progress toward inferring small numbers of pre-determined social roles and relationships from images and video, such as parent-child relationships in image collections~\cite{Gallagher,Wang2010,Murillo2012}; alliance clusters based on co-occurrences and shot analysis in movies~\cite{Ding2010,Ding2011}; leader-follower relationships in surveillance video~\cite{Yu2009,Zhang2011}; brides and grooms at weddings~\cite{FeiFeiRole2013}; player positions in football~\cite{LanSM12}; and speakers and listeners in meetings~\cite{meetingrolerecognition}. We seek to build on these successes by going beyond pre-determined roles and event types, automatically discovering the behavioral units that are relevant to different event types, and mapping actors to nodes in social networks that are multi-view and contains tens or hundreds of individuals.

%Seminal work has only appeared recently to infer elementary social attributes in each image such as `parents-children'\comment{from simple geometric configurations and appearance descriptors of faces or body parts} \cite{Gallagher,Wang2010,Murillo2012}, without connecting the image targets to nodes in the social network. `Alliance' clusters over the movie characters can be inferred from co-occurrences and shot analysis \cite{Ding2010,Ding2011}, the group leader can also be agglomerated from surveillance \cite{Yu2009,Zhang2011}, and roles in pre-defined social events (\emph{e.g.}, meetings \cite{meetingrolerecognition}, sport games \cite{LanSM12}, and weddings \cite{FeiFeiRole2013}) can be estimated within each video. These preliminary work are simplified cases of our expected outcome, which aims to explicitly mapping targets into nodes in the network of much larger scale. Instead of leveraging a few simple visual cues for inferring a few concepts, we will extract socially informative mid-level representations -- proxemes, aggregate repetitive occurrences of the proxemes from photo collections or surveillance networks \cite{CamNetRoy,CamNetSclaroff}, and develop systematic machines to estimate multi-view affinities between social members.



%work\cite{groupdet2013,LiIJCV2012,Li2010} are among the first set of work to distinguish salient activities from social groups. However, the most generic case we propose to address, is generic socially-informative proxemes both embedded in clutter and temporally localized, as a generalization of detecting `f-formation'~\cite{Cristani:fformation} and a counterpart to socially non-meaningful collective behaviors~\cite{Amer:group}.


%The earliest efforts were devoted to analyzing indoor meetings involving handfuls of participants \cite{GaticaPerez,McCowan:meeting}, where descriptors derived from tracking \cite{Smith:track} and pose estimation \cite{Ba:meeting} have been integrated in dynamic hidden Markov models (HMM)~\cite{Zhang:meeting}\todd{need more here. What was the output of their system? How is this different or similar to us? Has any work on meetings defined interaction categories, or recovered social network information from occurrences of these categories?}. Recent efforts have explored more complex visual environments, including the detection and recognition of a small number of pre-defined two-person interaction categories (e.g., hand-shaking, pushing, hugging)~\cite{UTdata}\todd{More here. Cite the related previous work in the CVPR submission.}; the detection of group conversations in a cocktail-party scenario; the detection of certain pre-defined categories of collective large-group behavior~\cite{Choi:context,Choi:recogtrack,Amer:group,Lan:Group}; and PI Li's work on segmenting offensive and defensive players in sports matches~\cite{LiIJCV2012}. The promising results from this assortment of sub-problems is part of what motivates us to pursue a large-scale effort to unify social and visual analysis more broadly, jointly considering all aspects of socialization in image and video data. \todd{It is important that we do not miss any related work in this paragraph. We need to be careful to include everything.}
