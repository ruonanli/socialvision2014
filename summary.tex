% !TEX root = SocialVision2012.tex
\pagestyle{empty}

\noindent\textsf{CIF: Small:}\vspace{0.9ex}\\
\noindent {\bf \textsf{\Large Foundations for Social Image Analytics: From Images to Social Graph}}

\vspace{1.0ex}
\noindent \textsf{\large \em Todd Zickler and Ruonan Li, Harvard University}
\vspace{2.0ex}



\noindent Image/video collections contain rich information about human interactions, relationships, and social structure: Whether it is a personal collection, a shared collection of an online community, or a collection from a surveillance network, any large set of visual observations of people over an extended period of time provides access to co-occurrence counts, relative face and body positions, gestures, expressions, as well as scene and event types, all of which provide information about identities, attributes, interactions, and relationships that are richer and more nuanced than those provided by textual signals like email logs, phone logs, and textual social media. This proposal aims to develop a computational foundation for image/video processing systems that can recognize social interactions and extract social relationship and social network from image/video collections. 

These systems will extract, from images/videos of human gatherings, information about the types and frequencies of social interactions that occur, and information about the social network represented as a social graph, which embeds the observed individuals. This information will undoubtedly be useful for machines, just as it is useful for humans when analyzing individuals within a social group from a distance, when deciding which colleagues to approach to enact policy changes, or when judging whether to alert authorities of a questionable interaction between strangers: If we had good tools for analyzing relationships and social networks from images, we could build better tools for discovering interaction patterns and inter-personal structures in active classrooms, sports matches, and other public spaces, and we could also build better human-computer interfaces, recommendation systems, video indexing methods, and security systems.

\boldstart{Intellectual Merit}. Achieving our goals requires coordinated innovation in two areas: semantic image processing for distilling social interactions, and multidimensional signal processing on graphs for establishing the social network.  For the former, we will introduce representations of interactions \comment{and relational attributes} based on proxemes---the interaction-analogy to phonemes in speech---that are modulated by the social roles and relationship between the actors. We will introduce models and similarity measures for proxemes as a foundation for the learning of proxeme vocabularies\comment{ from annotated data}, and the detection of proxeme occurrences in images and videos. In concert with these innovations, we will use spatio-temporal distributions of proxemes to define new functionals on social graphs (cell complexes) and develop tools to extremize these functionals. As a concrete example, one such extremization would simultaneously produce: 1) the number of unique individuals (0-cells, or nodes); 2) the identification of actors in each scene with their corresponding 0-cells, and personal attributes at each 0-cell, such as age and gender; 3) multi-dimensional signals of relational attributes at the 1-cells (edges), such as parent-child, friends, or colleagues, as well as communal attributes at higher-order cells.

\boldstart{Broader Impacts}. The proposed research will provide a foundation for socially-aware image processing systems that have the potential to transform security, surveillance, augmented reality, human computer interaction, human resource planning, operations research, and e-commerce. It will also benefit computational research in sociology, economics, and education by providing researchers with access to social patterns that are difficult or impossible to detect with the naked eye. The results of this research will be broadly disseminated by making datasets and software publicly available and by producing publications in refereed conferences and journals. The program will also impact education at the undergraduate and graduate levels, including giving undergraduates the opportunity to gain hands-on research experiences in participating in the research activity. 

\boldstart{Keywords:} Semantic Image processing; Social Network; Multi-dimensional Signals on Graphs

\comment{Three challenges combine to make this problem distinctive: 1) behaviors and interaction types must be inferred from imagery and are therefore noisy and ambiguous; 2) in addition, there is ambiguity in who is interacting because actor identities must also be inferred; and 3) like other network-inference problems, community structures exist at multiple scales, and measurements are incomplete because of unobserved co-occurrences. Meeting these challenges}  

%%%%%%%%% 2014 version %%%%%%%%%%%%%%%%


%\noindent This proposal aims to develop a computational foundation for computer vision systems that are socially-aware, in the sense of being able to recover from imagery information about social relationships and social networks. These systems will extract, from collections of images and videos of human gatherings, information about the types and frequencies of social interactions that occur, as well as information about the social network that embeds the observed individuals. This information will undoubtedly be useful for machines, just as it is useful for humans when analyzing individuals within a social group from a distance; when deciding which colleagues to approach to enact policy changes; or when judging whether to alert authorities of a questionable interaction between strangers.
%
%
%The proposed research will introduce computer vision as a new source of social network information, one that complements email logs, phone logs, web-based community connections, and so on. Vision is an important complement to these because it provides access to face positions, body poses, gestures, expressions, and scene types, all of which are difficult to observe by any other means but are critical signals for analyzing interactions and relationships.
%
%
%This research is timely because of the recent maturation of technologies for detecting and tracking the faces and bodies of people and recognizing their identities. A variety of practical systems exist, and by integrating them one can extract crude but useful descriptors of individual behaviors in the form of positions, trajectories, body poses, and so on. When image quality permits, these can also be augmented with more detailed per-individual descriptors computed from gestures  and facial expressions. In any case, with these descriptors an image (resp. video) can be abstracted as a collection of noisy per-agent detections (resp. trajectories) with accompanying (uncertain) behavior descriptors, and this is the abstraction on which the proposed framework will operate.

%\boldstart{Intellectual Merit}. Extracting social relationships and social networks from imagery poses several technical challenges, including: 1) representing interactions, and learning ``proxemes" --- interaction categories that are informative about relationships --- in semi-supervised and unsupervised settings; 2) detecting and recognizing proxemes in images or videos of large social gatherings; 3) inferring mixed-membership social networks from proxemes, accounting for noisy inputs and uncertain identities; and 4) collecting and disseminating meaningful datasets for development and evaluation. To address these challenges, the PIs will draw upon their expertise in analyzing individual and group behaviors; recognizing identities from faces embedded in a social network; creating public benchmark datasets; and enabling efficient learning by adapting recognition systems between different domains.






%%%%%%%%%%%%%%%%%%%%%%%%%%%%%%%%%

%\noindent This proposal aims to develop a computational foundation for computer vision systems that are socially-aware, in the sense of being able to recover from imagery information about social interactions and social relationships. These systems will extract, from collections of images and videos of human gatherings, information about the types and frequencies of social interactions that occur, as well as information about the social network that embeds the observed individuals. This information will undoubtedly be useful for machines, just as it is useful for humans when analyzing individuals within a social group from a distance; when deciding which colleagues to approach to enact policy changes; or when judging whether to alert authorities of a questionable interaction between strangers.


%We propose a framework to \emph{see} a social network - one that exploits images and videos and employs computer vision as a new `sensor' by which we aim to recover social relationships and attributes among individuals in our social communities. Contemporary research on social networks has been revealing us rich semantics by exploiting diverse `conventional social sensors' from textual/voice communications to online message sharing, but it has almost completely ignored visual media from abundant online photos to video volumes harvested by surveillance camera networks. Camera, as a new `community sensor', is much less intrusive than conventional social sensors, but exposes us the `real stories' about the community members from aside and remotely: Many of these stories may be too subtle for conventional approaches to grasp but visually sensible. Cameras, equipped with ever-increasing computer vision capabilities, produce socially-informaive data in a quantity thousand times of what conventional sensors do, but these data has largely remained unexplored for social networks. It is now our mission to introduce vision to social networks in this proposed research agenda.
%
%In companion, we propose to understand images and videos by incorporating the contextual knowledge provided by the online social connections where the images and videos are embedded, as well as to distill from images and videos new social semantics previously only under the attention of conventional sociology. This effort builds upon, but will go much beyond the preliminary attempt on recognizing faces and identities under social contexts: Face recognition only tells about \emph{who} are in the visual scene, but we argue that benefits from the socialized nature of media nowadays are not limited to the task of `who'. Knowing more about the social connections among the visual documents helps us to better understand \emph{what} the individuals are doing therein and \emph{where} they geographically are, and even enables us to predict more precisely \emph{whether} a particular event is to happen next.
%
%In fact, both the visual sensing of a social network and the socially assisted understanding of visual materials can benefit from cross-pollination: Visually sensed social ties provide more specific contextual evidences about who are more likely to interact in a new visual scene and what activities they are more likely to be engaged in, while watching the visual co-occurrences and co-activities of two community members may reflect more accurate connections between them that are not easily available from other resources. By socially-aware visual analytics, we mean a comprehensive theoretical and practical infrastructure that we expect to develop during the award period, with the two complementing modules of the visual sensing of a network and the socially assisted visual understanding well integrated.
%
%\boldstart{Intellectual Merit}. We propose a paradigm for socially-aware visual analytics, which systematically explores the interactions between visual information and social communities, yielding a new foundation for machine discovery of sociological knowledge and the potential for new perspectives and applications in visual understanding and computer vision. The paradigm includes new models and representations for networks arising from exploiting visual concepts in images and videos, and includes computational mechanisms to leverage socialized metadata for the new tasks in image and video analysis. The paradigm will, in particular, account for realistic situations in network sensing, such as multi-type overlapping communities and partial noisy observations, which has been largely overlooked by contemporary research, and all novel functionalities will build on mature elements in computer vision, such as face/human detection, recognition, and tracking, scene analysis, and event/activity interpretation, and will be deployed in the form of software and hardware.

%\boldstart{Intellectual Merit}. Extracting social interactions and social relationships from imagery poses several technical challenges, including: 1) learning models for social interaction categories, including in semi-superivsed and unsupervised settings; 2) detecting and recognizing interaction categories in long videos of large social gatherings; 3) inferring mixed-membership social networks from multiple noisy visual sources and uncertain identities; and 4) collecting and disseminating meaningful datasets for development and evaluation. To address these challenges, the PIs will draw upon their expertise in analyzing individual and group behaviors in videos; recognizing identities from faces embedded in a social network; creating public benchmark datasets; and enabling efficient learning by adapting recognition systems between different domains.

%
%\boldstart{Broader Impacts}. The primary application domain of the proposed research is online and networked visual media about humans, while it will be also applicable to broader visual materials involving other agents, such as historical image archives, scientific image collections, biological recordings, and so on. Socialized behaviors are prevalent in social creatures, and our research will shed lights on new approaches to automatic browse, index, and parse them. New insights will be drawn toward diverse disciplines spanning sociology, pedagogy, and statistics, which are still in their infancy in introducing automated approaches exploiting visual information. Besides the tremendous potentials of industrial interests and product implementation that do not need elaboration, the proposed interdisciplinary research will also prompt revolutions in educational programs providing next-generation students with more comprehensive knowledge and broader mastery of skills.
