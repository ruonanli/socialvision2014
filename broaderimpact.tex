% !TEX root = SocialVision2014.tex
\vspace{-8pt}
\section{Broader Impacts}
\label{sec:impacts}
\vspace{-8pt}
A framework that enables social understanding from visual data is critical for computer vision systems to better understand the world from its images. While these topics have received some attention from the research community, it is only now that that we have the data and computational resources to pursue it in any practical, real-world way. The accelerating development of computational infrastructure, the growth of digital video, and progress in detecting and tracking faces and people has finally made this research possible.

The proposed program will seize on this opportunity by concentrating our team's resources on the creation of a framework for social visual analytics. We will engage many others in its creation: students at carious levels, computer vision researchers, and the broader public. These are critical steps toward a future in which machines can better understand and interact with their environments; and they will usher in radical changes to security, personal robotics, augmented reality, human-computer interfaces, security, and video summarization, to name a few.

Our research will also transform computational research in sociology, economics, and education, by providing access to social patterns that are difficult or impossible to detect with the naked eye. Traditionally, the analysis of interactions has required immense effort by social science experts. Collections of imagery must be viewed repeatedly before salient and meaningful proxemes can be defined, and their occurrences\comment{ in new videos} must be painstakingly and redundantly coded by trained human observers. Our research will transform machine analyses in these disciplines by greatly simplifying the process of defining proxemes, by automating the detection of instances of the proxemes, and by automatically reconstructing social relationships from these detections. This will enable analyses in a much wider set of environments and over transformative periods of time.

Our results will be broadly disseminated in three ways. First, they will be distributed by publications in refereed conferences and journals and by lectures at other institutions. Second, datasets and source code will be posted online so that they can be easily adopted and extended (see Section \ref{sec:sys} and the Data Management Plan). Third, they will be shared through challenge problem workshops on social visual analytics, held in conjunction with the major vision conferences\comment{ with the small amount of required funding provided by sponsorship agreements}. The research team has an established track record in this type of outreach, with PI Zickler having organized tutorials in vision and graphics (ICCV 2007, SIGGRAPH 2008) and both PIs having organized workshops on computer vision (CRICV 2009, 2010, CPCV 2011, 2013, VISDA 2013).

\vspace{-8pt}
\section{Curriculum Development Activities}
\label{sec:curriculum}
\vspace{-8pt}
The proposed research program will play an important role in education at both the undergraduate and graduate levels. The fundamental results of this research will be incorporated into the curriculum of a popular computer vision class at Harvard University, \emph{CS283: Computer Vision}, which serves both senior undergraduate students and graduate students at Harvard University. Undergraduates at the junior and senior levels will be also be given opportunities for hands-on research experience in developing components of the proposed framework. This includes extra-curricular and summer research internships that will be funded by future NSF REU supplements, as well as those funded by our institution's many internal funding mechanisms, including the Harvard College Program for Research in Science and Engineering; the Herchel Smith-Harvard Undergraduate Science Research Program; the Harvard College Research Program; and the Mellon Mays Undergraduate Fellowship Program, which is is specifically targeted at underrepresented minorities. Finally, funding will be used to support one full-time PhD student and one part-time PhD student, and these students will be an integral part of both research and educational endeavors in this project. They will participate in all stages, including the design of equipment and algorithms, curricular development, and presentation of results at major conferences.

